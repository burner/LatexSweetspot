\documentclass[a4paper,openright,BCOR20mm]{scrbook}

\usepackage{color}
\usepackage{listings}
\usepackage{sweetspot}
\usepackage{subfiles}
\usepackage{subpackages}

\lstdefinelanguage{d}{
	morekeywords={abstract,alias,align,asm,assert,auto,body,bool,break,byte,
		case,cast,catch,cdouble,cent,cfloat,char,class,const,continue,creal,
		dchar,debug,default,delegate,delete,deprecated,do,double,else,enum,
		export,extern,false,final,finally,float,for,foreach,foreach_reverse,
		function,goto,idouble,if,ifloat,immutable,import,in,inout,int,
		interface,invariant,ireal,is,lazy,long,macro,mixin,module,new,nothrow,
		null,out,override,package,pragma,private,protected,public,pure,real,
		ref,return,scope,shared,short,static,struct,string,wstring,dstring,
		super,switch,synchronized,template,this,throw,true,try,typedef,typeid,
		typeof,ubyte,ucent,uint,ulong,union,unittest,ushort,version,void,
		volatile,wchar,while,with,size_t,hash_t
	},
	sensitive=True,
	morecomment=[l]{//},
	morecomment=[s]{/\*}{\*/},
	morestring=[b]"
}

\lstset{
	language=d,
	basicstyle=\footnotesize\ttfamily, % Standardschrift
	numbers=left,               % Ort der Zeilennummern
	numberstyle=\tiny,          % Stil der Zeilennummern
	%stepnumber=2,               % Abstand zwischen den Zeilennummern
	captionpos=b,
	numbersep=5pt,              % Abstand der Nummern zum Text
	tabsize=4,                  % Groesse von Tabs
	extendedchars=true,         %
	breaklines=true,            % Zeilen werden Umgebrochen
	showspaces=false,           % Leerzeichen anzeigen ?
	showtabs=false,             % Tabs anzeigen ?
	%frame=b,
	rulecolor=\color[cmyk]{0.0, 0.0, 0.0, 0.316},
	xleftmargin=17pt,
	%xleftmargin=0pt,
	framexleftmargin=17pt,
	framexrightmargin=5pt,
	framexbottommargin=4pt,
	keywordstyle=\color{blue},          % keyword style
	commentstyle=\color{dkgreen},       % comment style
	stringstyle=\color{mauve},         % string literal style
	showstringspaces=false,    % Leerzeichen in Strings anzeigen ?        
	escapeinside={/+}{+/},
	title=\lstname
}

\usepackage[english]{babel}
\usepackage[utf8]{inputenc}
\usepackage[
    backend=biber,
    style=authoryear-icomp,
    sortlocale=de_DE,
    natbib=true,
    url=false, 
    doi=true,
    style=numeric,
    eprint=false,
	backref
]{biblatex}
\addbibresource{biblio.bib}

%\usepackage[miniindex]{authorindex}

\input{glossary.gls}
\input{acronyms.gls}

%\makeindex
\makenoidxglossaries
\usepackage[xindy]{imakeidx}

\author{Robert burner Schadek}
\title{D Sweetspot}
\subtitle{Using D's Standard Library Phobos to Create the Perfect XML Parser}

\begin{document}
\maketitle
\tableofcontents

\subfile{introduction}

\chapter{Getting Started}
In this book the D standard library Phobos is used to create a replacement for
the XML facilities given in std.xml;
Performance is a crucial part when it comes to the development of an XML parser.
Nobody wants to use a slow XML parser.
To keep an eye on the performance of the parser we need test data to
benchmark.
Good test data is hard to come by, therefore we will at least partially,
create our own.
This listing allows us to get an idea of what data we can accept.
We will use std.csv to parse that data and then use this data to generate
\G{xml} files of different sizes.
\lstinputlisting[label=lst:csvexample1_1,caption={struct to store csv data in.},lastline=14]{csvexample1.d}
The lines \Rl{lst:csvexample1_beginmember} to \Rl{lst:csvexample1_endmember}
of Listing \Rl{lst:csvexample1_1} show the names of the values present in the csv
file.
\G{ctfe}

%
% APPENDIX
%

\chapter*{Appendix}
\appendix
\phantomsection \label{appendix}
\addcontentsline{toc}{chapter}{Appendix}
\cleardoublepage

\phantomsection\label{glossary}
\addcontentsline{toc}{section}{Glossary}
\printnoidxglossary

\cleardoublepage
\phantomsection\label{acronyms}
\addcontentsline{toc}{section}{Acronyms}
\printnoidxglossary[type=acronym]

\cleardoublepage
\phantomsection\label{figures}
\addcontentsline{toc}{section}{Figures}
\listoffigures
 
\cleardoublepage
\phantomsection\label{tables}
\addcontentsline{toc}{section}{Tables}
\listoftables

\cleardoublepage
\phantomsection\label{listings}
\addcontentsline{toc}{section}{Listings}
\lstlistoflistings

\cleardoublepage
\phantomsection\label{bibliography}
\addcontentsline{toc}{section}{Bibliography}
%\bibliographystyle{alpha}
%\bibliography{biblio}
\printbibliography

%\chapter*{Appearances of Authors}
%\printauthorindex

\end{document}
